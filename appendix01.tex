\chapter{Summation}\label{app:appendix01}

This appendix offers methods for evaluating summations, which occur frequently in the analysis of algorithms. Many
of the formulas here appear in any calculus text, but you will find it convenient to have these methods compiled in one place. The content of this appendix is based on the book \textit{Introduction to Algorithms}, (\cite{clrs4}). The Appendix is concluded with a useful cheat sheet for time complixity analysis.

When an algorithm contains an iterative control construct such as a \While or \For loop, you can express its running time as the sum of the times spent on each execution of the body of the loop. For example, in Section \ref{sec:loop-analysis} we argued that the $i$'th iteration of $\proc{Quadratic-Sum}$ took time proportional to $i$ in the worst case. Adding up the time spent on each iteration produced the summation (or series) $\sum\limits_{i=1}^n i$. Evaluating this summation resulted in a bound of $\mathcal{O}(n^2)$ on the worst-case running time of the algorithm. This example illustrates why you should know how to manipulate summations. Before going into details, we provide an overview of the most important summations and their closed forms.

\begin{table}[h!]
    \centering
    \renewcommand{\arraystretch}{2}
    \begin{tabular}{|c|c|}
    \hline
    \textbf{Sum} & \textbf{Closed Form} \\ \hline
    $\displaystyle \sum_{k=0}^n ar^k \, (r \neq 1)$ & $\displaystyle \frac{ar^{n+1} - a}{r - 1}$ \\ \hline
    $\displaystyle \sum_{k=1}^n k$ & $\displaystyle \frac{n(n+1)}{2}$ \\ \hline
    $\displaystyle \sum_{k=1}^n k^2$ & $\displaystyle \frac{n(n+1)(2n+1)}{6}$ \\ \hline
    $\displaystyle \sum_{k=1}^n k^3$ & $\displaystyle \frac{n^2(n+1)^2}{4}$ \\ \hline
    $\displaystyle \sum_{k=0}^n x^k \, (x \neq 1)$ & $\displaystyle \frac{x^{n+1} - 1}{x - 1}$ \\ \hline
    $\displaystyle \sum_{i=1,2,4,\ldots,n} i$ & $\displaystyle 2n - 1$ \\ \hline
    $\displaystyle \sum_{k=0}^\infty x^k, \, |x| < 1$ & $\displaystyle \frac{1}{1 - x}$ \\ \hline
    $\displaystyle \sum_{k=1}^\infty kx^{k-1}, \, |x| < 1$ & $\displaystyle \frac{1}{(1 - x)^2}$ \\ \hline
    $\displaystyle \sum_{k=1}^n \frac{1}{k}$ & $\displaystyle \ln n + O(1)$ \\ \hline
    $\displaystyle \sum_{k=1}^n (a + bk)$ & $\displaystyle a n + b \frac{n(n+1)}{2}$ \\ \hline  
    \end{tabular}
    \caption{Some Useful Summation Formulae}
\end{table}


\section*{Summation Notation}

Consider a sequence of numbers \( a_1, a_2, \ldots, a_n \), where \( n \) is a nonnegative integer. The sum of this sequence, \( a_1 + a_2 + \dots + a_n \), can be represented by the notation \( \sum_{k=1}^n a_k \). When \( n = 0 \), this summation is defined to have a value of 0. The result of a finite sum is always well-defined, and the order in which the terms are summed does not affect the final value.

For an infinite sequence \( a_1, a_2, \ldots \) of numbers, we represent their infinite sum \( a_1 + a_2 + \dots \) by the notation \( \sum_{k=1}^{\infty} a_k \), which corresponds to \( \lim_{n \rightarrow \infty} \sum_{k=1}^n a_k \). If this limit exists, the series is said to converge; otherwise, it diverges. Unlike finite sums, the terms of a convergent series cannot necessarily be rearranged without affecting the outcome. However, in an absolutely convergent series—one where \( \sum_{k=1}^{\infty} |a_k| \) also converges — the terms can be reordered without changing the sum.

\[
    \sum_{i=i_0}^n c a_i=c \sum_{i=i_0}^n a_i
\]

where $c$ is any number. So, we can factor constants out of a summation.

Similarly, we can split a summation into two summations:

\[
\sum_{i=i_0}^n\left(a_i \pm b_i\right)=\sum_{i=i_0}^n a_i \pm \sum_{i=i_0}^n b_i
\]

Note that we started the series at $i_0$ to denote the fact that they can start at any value of $i$ that we need them to. Also note that while we can break up sums and differences as we did above we can't do the same thing for products and quotients. In other words,

\[
    \sum_{i=i_0}^n\left(a_i b_i\right) \neq\left(\sum_{i=i_0}^n a_i\right)\left(\sum_{i=i_0}^n b_i\right) \qquad \text{and} \qquad \sum_{i=i_0}^n \frac{a_i}{b_i} \neq \frac{\sum_{i=i_0}^n a_i}{\sum_{i=i_0}^n b_i}.
\]

\section*{Linearity of Summation}
For any real number $c$ and any finite sequences $a_1, a_2, \ldots, a_n$ and $b_1, b_2, \ldots, b_n$,

\[
    \sum_{k=1}^n\left(c a_k+b_k\right)=c \sum_{k=1}^n a_k+\sum_{k=1}^n b_k .
\]

The linearity property also applies to infinite convergent series.

The linearity property applies to summations incorporating asymptotic notation. For example,

\[
    \sum_{k=1}^n \Theta(f(k))=\Theta\left(\sum_{k=1}^n f(k)\right) .
\]

In this equation, the $\Theta$-notation on the left-hand side applies to the variable $k$, but on the right-hand side, it applies to $n$. Such manipulations also apply to infinite convergent series.

\section*{Arithmetic Series}
The summation


\[
        \sum_{k=1}^n k=1+2+\cdots+n \label{eq:arithmetic-series}
\]

is an arithmetic series and has the value


\begin{align}
    \sum_{k=1}^n k & =\frac{n(n+1)}{2} \label{eq:sum-of-natural-numbers}  \\
    & =\Theta\left(n^2\right) . \label{eq:arithmetic-series-asymptotic}
\end{align}

A general arithmetic series includes an additive constant $a \geq 0$ and a constant coefficient $b>0$ in each term, but has the same total asymptotically:

\begin{equation}
    \sum_{k=1}^n(a+b k)=\Theta\left(n^2\right) . \label{eq:general-arithmetic-series}
    \end{equation}

\section*{Sums of Squares and Cubes}
The following formulas apply to summations of squares and cubes:

    \begin{equation}
    \sum_{k=0}^n k^2=\frac{n(n+1)(2 n+1)}{6}, \label{eq:sum-of-squares}
    \end{equation}

    \begin{equation}
        \sum_{k=0}^n k^3=\frac{n^2(n+1)^2}{4} . \label{eq:sum-of-cubes}
        \end{equation}

        \section*{Geometric Series}
        For real $x \neq 1$, the summation
        
        \[
            \sum_{k=0}^n x^k=1+x+x^2+\cdots+x^n
        \]
        
        is a geometric series and has the value
        
        \begin{equation}
            \sum_{k=0}^n x^k=\frac{x^{n+1}-1}{x-1} . \label{eq:geometric-series}
        \end{equation}
        
        The infinite decreasing geometric series occurs when the summation is infinite and $|x|<1$:
        
        \begin{equation}
            \sum_{k=0}^{\infty} x^k=\frac{1}{1-x} . \label{eq:infinite-geometric-series}
        \end{equation}
        
        If we assume that $0^0=1$, these formulas apply even when $x=0$.
        
        A \textbf{special} case of the geometric series arises when the terms form a progression of powers of 2. That is, the terms are \(1, 2, 4, 8, \ldots, n\). In this case:
        \[
            \sum_{i=1,2,4,\ldots,n} i = 1 + 2 + 4 + 8 + \cdots + n.
        \]
        
        This series can be derived from the general formula for a geometric series. The general form of a geometric series is:
        \[
            S = a + ar + ar^2 + \cdots + ar^{k-1},
        \]
        where:
        \begin{itemize}
            \item \(a\) is the first term (\(a=1\) in this case),
            \item \(r\) is the common ratio (\(r=2\) here),
            \item \(k\) is the number of terms in the series.
        \end{itemize}
        The number of terms, \(k\), is determined by how many times the progression \(1, 2, 4, \ldots, n\) doubles until reaching \(n\). Since \(n = 2^{k-1}\), we have \(k = \log_2(n) + 1\).
        
        Using the geometric series formula \eqref{eq:geometric-series}:
        \[
            \sum_{i=1,2,4,\ldots,n} i = \frac{2^k - 1}{2 - 1} = 2^k - 1.
        \]
        
        Substituting \(k = \log_2(n) + 1\):
        \[
            \sum_{i=1,2,4,\ldots,n} i = 2^{\log_2(n)+1} - 1 = 2n - 1.
        \]
        
        Thus, the sum of the powers of 2 up to \(n\) is:
        \[
            \sum_{i=1,2,4,\ldots,n} i = 2n - 1.
        \]
        

\section*{Harmonic Series}
For positive integers $n$, the $n$th harmonic number is

\begin{align}
    H_n & = 1 + \frac{1}{2} + \frac{1}{3} + \frac{1}{4} + \cdots + \frac{1}{n} \nonumber \\
        & = \sum_{k=1}^n \frac{1}{k} \label{eq:harmonic_sum} \\
        & = \ln n + O(1) . \label{eq:approximation}
\end{align}

\section*{Integrating and Differentiating}
Integrating or differentiating the formulas above yields additional formulas. For example, differentiating both sides of the infinite geometric series (\ref{eq:infinite-geometric-series}) and multiplying by $x$ gives


\begin{equation}
    \begin{aligned}
    & \sum_{k=0}^{\infty} k x^k=\frac{x}{(1-x)^2} \\ 
    & \text { for }|x|<1 . \label{eq:diff-geometric-series}
    \end{aligned}
\end{equation}

\section*{Telescoping Series}

For any sequence $a_0, a_1, \ldots, a_n$

\begin{equation}
    \sum_{k=1}^n\left(a_k-a_{k-1}\right)=a_n-a_0, \label{eq:telescoping-series}
\end{equation}

since each of the terms $a_1, a_2, \ldots, a_{n-1}$ is added in exactly once and subtracted out exactly once. We say that the sum telescopes. Similarly,

\[
    \sum_{k=0}^{n-1}\left(a_k-a_{k+1}\right)=a_0-a_n .
\]

As an example of a telescoping sum, consider the series

\[
\sum_{k=1}^{n-1} \frac{1}{k(k+1)}
\]

Rewriting each term as

\[
\frac{1}{k(k+1)}=\frac{1}{k}-\frac{1}{k+1},
\]

gives

\begin{align}
\sum_{k-1}^{n-1} \frac{1}{k(k+1)} & =\sum_{k-1}^{n-1}\left(\frac{1}{k}-\frac{1}{k+1}\right) \nonumber \\
& =1-\frac{1}{n} . \label{eq:telescoping-example}
\end{align}

\section*{Reindexing Sums}
A series can sometimes be simplified by changing its index, often reversing the order of summation. Consider the series $\sum_{k=0}^n a_{n-k}$. Because the terms in this summation are $a_n, a_{n-1}, \ldots, a_0$, we can reverse the order of indices by letting $j=n-k$ and rewrite this summation as

$$
\sum_{k=0}^n a_{n-k}=\sum_{l=0}^n a_j
$$


Generally, if the summation index appears in the body of the sum with a minus sign, it's worth thinking about reindexing.

As an example, consider the summation

\begin{equation}
    \sum_{k=1}^n \frac{1}{n-k+1} . \label{eq:reindexing-example}
\end{equation}


The index $k$ appears with a negative sign in $\dfrac{1}{(n-k+1)}$. And indeed, we can simplify this summation, this time setting $j=n-k+1$, yielding

\begin{equation}
    \sum_{k=1}^n \frac{1}{n-k+1}=\sum_{i=1}^n \frac{1}{j}, \label{eq:reindexing-example-solution} 
\end{equation}

which is just the harmonic series (\ref{eq:harmonic_sum}).

\section*{Products}
The finite product $a_1 a_2 \ldots a_n$ can be expressed as

\[
\prod_{k=1}^n a_k .
\]


If $n=0$, the value of the product is defined to be 1 . You can convert a formula with a product to a formula with a summation by using the identity

\[
\log \left(\prod_{k=1}^n a_k\right)=\sum_{k=1}^n \log a_k .
\]

\section*{Cheat Sheet for Time Complexity Analysis}

This section provides essential formulas and shortcuts for asymptotic analysis and algorithm evaluation. These complement the detailed summation methods in this appendix.

\begin{itemize}
    \item \textbf{Logarithmic Factorial Approximation (Stirling's Approximation):} 
    \[
    \log(n!) \approx n \log n - n.
    \]
 
    \item \textbf{Logarithmic Exponent Rule:} 
    \[
    2^{\log_2 n} = n.
    \]

    \item \textbf{Base as a Power with Logarithmic Exponent:} 
    If the base is \( a^b \) and the exponent involves \( \log_a \), then:
    \[
    (a^b)^{\log_a(n^c)} = n^{bc}.
    \]
    
    \textbf{Example:} \( 4^{\log_2(n^2)} = n^4 \) (since \( 4 = 2^2 \)).
    
    \item \textbf{Power of a Logarithm Rule:} 
    \[
    \log(n^k) = k \log n, \quad \text{for } k > 0.
    \]
    
    \item \textbf{Logarithmic Growth for Powers of 2:} 
    For \( n = 2^k \),
    \[
    k = \log_2 n.
    \]
    
    \item \textbf{Simplifying Logarithmic Expressions:} 
    \[
    \log_a n = \frac{\log_b n}{\log_b a}, \quad \text{for } a, b > 0.
    \]
    
    \item \textbf{Common Asymptotic Comparisons:} 
    \[
    n^c \ll c^n \quad \text{for any constant } c > 1.
    \]
    
    \item \textbf{Factorial Asymptotics (Expanded Stirling's Approximation):} 
    Using Stirling's formula, the growth of \( n! \) is approximately:
    \[
    n! = \Theta\left(\sqrt{2\pi n} \left(\frac{n}{e}\right)^n\right).
    \]
    
    \item \textbf{Sum of Powers of 2:} 
    If the terms in a series form powers of 2 (e.g., \( 1, 2, 4, 8, \ldots, n \)), the sum is:
    \[
    \sum_{i=1,2,4,\ldots,n} i = 2n - 1.
    \]

\end{itemize}
