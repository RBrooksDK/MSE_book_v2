\documentclass{article}
\usepackage{circuitikz}
\usetikzlibrary{circuits.logic.US}

\begin{document}
\begin{circuitikz} \draw
    (0,0) node[and gate US, draw, logic gate inputs=nn] (and1) {}
    (and1.input 1) -- ++(-1,0) node[left] {$x$}
    (and1.input 2) -- ++(-1,0) node[left] {$y$}

    (0,-2) node[not gate US, draw, anchor=input] (not1) {}
    (not1.input) -- ++(-1,0) node[left] {$x$}

    (not1.output) -- ++(1,0) |- (2,-2) node[and gate US, draw, logic gate inputs=nn] (and2) {}
    (and2.input 2) -- ++(-1,0) node[left] {$y$}

    (and1.output) -- ++(1,0) node[or gate US, draw, anchor=input 1] (or1) {}
    (and2.output) -- ++(1,0) |- (or1.input 2)

    (or1.output) -- ++(1,0) node[right] {$xy + \overline{x}y$};
\end{circuitikz}
\end{document}
