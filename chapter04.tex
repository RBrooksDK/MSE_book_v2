\chapter{Combinatorics and Probability Theory}
\label{chap:ch4}
Imagine you are tasked with forming teams of 3 for a semester project in a class of 45 students. Initially, the order in which you choose the team members does not matter, so you are just concerned with combinations. The number of ways to form a team of 3 from 45 students comes out to 14,190 possibilities!

The following semester introduces the students to Scrum project management, where each team must have three specific roles: Scrum Master, Product Owner, and Development Team. This small change, specifying roles, suddenly transforms the problem from a simple \textit{combination} into a \textit{permutation}. Now, the number of possible ways to assign these roles leaps to 85,140!

Frustrated by the sheer number of options, the 45 students throw a party to relax. Being well-mannered, they decide that everyone should shake hands with every other person exactly once. After a few minutes, they calculate the total number of handshakes — 990. The students are once again surprised by how something as simple as shaking hands can add up so quickly.

As the night progresses, one student proposes a fun game — a random drawing for five door prizes, each unique. With 45 students in attendance and only five prizes available, the chance of winning nothing becomes a concerning 89 per cent. The students quickly realize that the odds are not in their favor.

Not ready to give up on their luck, a smaller group decides to flip a coin 10 times, with the hope of landing exactly five heads to win the game. However, when they learn that the probability of this happening is only about 25 per cent, their spirits dampen further.

The students conclude that rather than relying on chance, it is time to dive deeper into understanding combinatorics and probability theory. Armed with this knowledge, they can better predict outcomes and avoid future disappointments at both parties and project planning.

\section{Sample Space and Events}
\label{sec:sample_space}
A \textbf{random experiment} is one that can lead to different outcomes, even when repeated under the same conditions. This randomness is a fundamental aspect of many engineering tasks.

\begin{definition}{Random Experiment}
    A random experiment is one that can give different results, even if you do everything the same each time.
\end{definition}

To model and analyze a random experiment, it is crucial to understand the set of possible outcomes that can occur. In probability theory, this set is called the \textbf{sample space}, denoted by \( S \). A sample space can be either \textbf{discrete} (consisting of a finite or countably infinite set of outcomes) or \textbf{continuous} (containing an interval of real numbers). An \textbf{outcome} is a single possible result of the random experiment, and an \textbf{event} is any subset of the sample space.

\begin{example}{Network Latency} \\
Consider an experiment where you measure the latency of data packets in a network. The sample space can be defined based on the type of measurements:
\begin{itemize}
    \item If latency is measured as a positive real number, the sample space is continuous: $S = \{ x \mid x > 0 \}$.
    \item If it is known that latency ranges between 10 and 100 milliseconds, the sample space can be refined to: $S = \{ x \mid 10 \leq x \leq 100 \}$.
    \item If the objective is to categorize latency as low, medium, or high, the sample space becomes discrete: $S = \{\text{low}, \text{medium}, \text{high}\}$.
\end{itemize}
\end{example}

\begin{definition}{Sample Spaces, Outcomes, and Events}
    \textbf{Sample Space:} The set of all possible outcomes of a random experiment, denoted by \( S \).
    
    \textbf{Outcome:} A single possible result of a random experiment.
    
    \textbf{Event:} Any subset of the sample space, which may consist of one or more outcomes.
\end{definition}

\section{Types of Events in Probability Theory}
\label{subsec:types_events}
The study of probability and set theory are deeply connected. In probability, we work with events, which are subsets of a sample space. The operations on these events correspond directly to the set and Boolean algebra operators we have already seen.

\subsection*{Union and Logical \texttt{OR}}
The \textbf{union} of two events \( A \) and \( B \), denoted $A \cup B$, represents the event that at least one of the events occurs. This corresponds to the logical \texttt{OR} operator.

\subsection*{Intersection and Logical \texttt{AND}}
The \textbf{intersection} of two events \( A \) and \( B \), denoted $A \cap B$, represents the event that both \( A \) and \( B \) occur simultaneously. This corresponds to the logical \texttt{AND} operator.

\subsection*{Complement and Logical \texttt{NOT}}
The \textbf{complement} of an event \( A \), denoted $A^C$, represents all outcomes in the sample space that are not in \( A \). This operation is analogous to the logical \texttt{NOT} operator.

\begin{remark}
    The complement of event $A$ is most commonly written as $A^C$. You may also encounter the notations $\overline{A}$ or $A'$ in other literature. This book will consistently use $A^C$.
\end{remark}

\subsection*{Additional Events}
\begin{itemize}
    \item \textbf{Difference (\( A - B \))}: The event that occurs if \( A \) happens but \( B \) does not.
    \item \textbf{Symmetric Difference (\( A \triangle B \))}: The event that occurs if exactly one of \( A \) or \( B \) happens, but not both. This corresponds to the logical \texttt{XOR} operator.
    \item \textbf{Mutually Exclusive Events (Disjoint)}: Two events are mutually exclusive if they cannot occur at the same time ($A \cap B = \emptyset$).
\end{itemize}

% Combined Venn Diagrams in a 3x2 Layout
\begin{figure}[htbp]
    \centering
    
    % --- ROW 1 ---
    \begin{subfigure}[b]{0.45\textwidth}
        \centering
        \begin{tikzpicture}
            % Using \firstcircle and \secondcircle defined in preamble
            \draw[outline] (-2,-2) rectangle (4,2);
            \fill[filled] \firstcircle \secondcircle;
            \draw[outline] \firstcircle node {$A$};
            \draw[outline] \secondcircle node {$B$};
        \end{tikzpicture}
        \caption{Union ($A \cup B$)}
    \end{subfigure}
    \hfill % Pushes the two subfigures apart to fill the line
    \begin{subfigure}[b]{0.45\textwidth}
        \centering
        \begin{tikzpicture}
            \draw[outline] (-2,-2) rectangle (4,2);
            \begin{scope}
                \clip \firstcircle;
                \fill[filled] \secondcircle;
            \end{scope}
            \draw[outline] \firstcircle node {$A$};
            \draw[outline] \secondcircle node {$B$};
        \end{tikzpicture}
        \caption{Intersection ($A \cap B$)}
    \end{subfigure}

    \vspace{1em} % Adds some vertical space between rows

    % --- ROW 2 ---
    \begin{subfigure}[b]{0.45\textwidth}
        \centering
        \begin{tikzpicture}
            \draw[outline] (-2.5, -2) rectangle (2.5, 2);
            \fill[filled] (-2.5, -2) rectangle (2.5, 2);
            \fill[white] (0,0) circle (1.5cm); % Using explicit circle for centering
            \draw[outline] (0,0) circle (1.5cm) node {$A$};
        \end{tikzpicture}
        \caption{Complement ($A^C$)}
    \end{subfigure}
    \hfill
    \begin{subfigure}[b]{0.45\textwidth}
        \centering
        \begin{tikzpicture}
            \draw[outline] (-2,-2) rectangle (4,2);
            \begin{scope}
                 \clip \firstcircle;
                 \fill[filled, even odd rule] \firstcircle \secondcircle;
            \end{scope}
            \draw[outline] \firstcircle node {$A$};
            \draw[outline] \secondcircle node {$B$};
        \end{tikzpicture}
        \caption{Difference ($A - B$)}
    \end{subfigure}

    \vspace{1em} % Adds some vertical space between rows

    % --- ROW 3 ---
    \begin{subfigure}[b]{0.45\textwidth}
        \centering
        \begin{tikzpicture}
            \draw[outline] (-2,-2) rectangle (4,2);
            \fill[filled, even odd rule] \firstcircle \secondcircle;
            \draw[outline] \firstcircle node {$A$};
            \draw[outline] \secondcircle node {$B$};
        \end{tikzpicture}
        \caption{Symmetric Difference ($A \triangle B$)}
    \end{subfigure}
    \hfill
    \begin{subfigure}[b]{0.45\textwidth}
        \centering
        \begin{tikzpicture}
            % Rewritten for consistency
            \draw[outline] (-3,-2) rectangle (3,2);
            \draw[outline] (-1,0) circle (1.2cm) node {$A$};
            \draw[outline] (1.5,0) circle (1.2cm) node {$B$};
        \end{tikzpicture}
        \caption{Mutually Exclusive}
    \end{subfigure}

    \caption{Venn diagrams illustrating set operations.}
    \label{fig:combined_venn_3x2}
\end{figure}

\newpage

The table below summarizes the direct correspondence between the operators used in Set Theory, Boolean Algebra, and Logic.

\begin{table}[htbp]
    \centering
    \renewcommand{\arraystretch}{1.5}
    \begin{tabular}{|c|c|c|c|}
    \hline
    \textbf{Operation} & \textbf{Set Theory} & \textbf{Boolean Algebra} & \textbf{Logic} \\
    \hline
    \texttt{NOT} & $A^C$ & $\overline{x}$ & $\neg x$ \\
    \hline
    \texttt{OR} & $\cup$ & $+$ & $\vee$ \\
    \hline
    \texttt{AND} & $\cap$ & $\cdot$ & $\wedge$ \\
    \hline
    \texttt{NAND} & $(A \cap B)^C$ & $\overline{x \cdot y}$ & $\neg (x \wedge y)$\\
    \hline
    \texttt{NOR} & $(A \cup B)^C$ & $\overline{x + y}$ & $\neg (x \vee y)$ \\
    \hline
    \texttt{XOR} & $A \triangle B$ & $x \oplus y$ & $(x \wedge \neg y) \vee (\neg x \wedge y)$ \\
    \hline
    Difference & $A - B$ & $x \cdot \overline{y}$ & $x \wedge \neg y$ \\
    \hline
    \end{tabular}
    \caption{Comparison of Operators in Set Theory, Boolean Algebra, and Logic}
\end{table}

\section{Counting Principles}
In many problems, determining the number of ways certain events can occur is essential. Counting techniques, such as permutations and combinations, help us quantify these possibilities systematically.

\subsection*{Multiplication Rule}
The most basic counting principle is the \textbf{multiplication rule}.
\begin{theorem}[Multiplication Rule]
    If an operation consists of a sequence of $k$ steps, and there are $n_1$ ways to do the first step, $n_2$ ways to do the second step, and so on, then the total number of ways to complete the operation is $n_1 \times n_2 \times \cdots \times n_k$.
\end{theorem}

\subsection*{Permutations (Order Matters)}
A permutation is an arrangement of objects in a specific order.
\begin{definition}[Permutations of Subsets]
    The number of permutations of $r$ elements selected from a set of $n$ elements is
    \[ P_r^n = \frac{n!}{(n-r)!} \]
\end{definition}

\begin{example}
    There are 10 entries in a contest. The 1st, 2nd, and 3rd place prizes are awarded. How many possible results are there?
\end{example}
\begin{solution}
    The order of selection matters, so we use permutations. The number of ways to award the prizes is the number of permutations of 3 objects selected from 10:
    \[ P_3^{10} = \frac{10!}{(10-3)!} = \frac{10!}{7!} = 10 \times 9 \times 8 = 720 \]
\end{solution}

\subsection*{Combinations (Order Does Not Matter)}
A combination is a selection of objects where the order does not matter.
\begin{definition}[Combinations]
    The number of combinations of $r$ elements selected from a set of $n$ elements is given by
    \[ C_r^n = \binom{n}{r} = \frac{n!}{r!(n-r)!} \]
\end{definition}

The formula is derived by taking the number of permutations ($P_r^n$) and dividing by $r!$, which is the number of ways to arrange the $r$ selected items. This division effectively removes the duplicates created by ordering.

\begin{example}
    A circuit board has four locations. If three identical components are to be placed on the board, how many different designs are possible?
\end{example}
\begin{solution}
    Since the components are identical, the order of placement does not matter. We need to choose 3 locations out of 4.
    \[ \binom{4}{3} = \frac{4!}{3!(4-3)!} = \frac{4!}{3! \cdot 1!} = \frac{4}{1} = 4 \]
    There are 4 possible designs.
\end{solution}

\begin{example}
    Maria has three tickets for a concert and wants to invite two of her four friends (Ann, Beth, Chris, Dave). How many ways can she choose 2 friends?
\end{example}
\begin{solution}
    The order in which she chooses her friends does not matter, so we use combinations.
    \[ \binom{4}{2} = \frac{4!}{2!(4-2)!} = \frac{4 \cdot 3}{2 \cdot 1} = 6 \]
    There are 6 possible pairs of friends she can invite.
\end{solution}

\section{Probability Basics}
Probability quantifies the likelihood of an outcome. When all outcomes in a finite sample space are equally likely, we can use a straightforward formula to calculate the probability of an event.

\begin{definition}[Probability with Equally Likely Outcomes]
    For a random experiment where all outcomes in the finite sample space $S$ are equally likely, the probability of an event $A$ is the ratio of the number of favorable outcomes to the total number of outcomes.
    \[ P(A) = \frac{\text{Number of outcomes in } A}{\text{Total number of outcomes in } S} = \frac{|A|}{|S|} \]
\end{definition}

All probability assignments, regardless of the scenario, must adhere to three fundamental rules known as the Axioms of Probability.
\begin{axiom}[Axioms of Probability]
    \begin{itemize}
        \item \textbf{Axiom 1:} For any event $A$, $0 \leq P(A) \leq 1$.
        \item \textbf{Axiom 2:} $P(S) = 1$.
        \item \textbf{Axiom 3:} If $A_1, A_2, \dots$ are disjoint events, then $P(A_1 \cup A_2 \cup \dots) = P(A_1) + P(A_2) + \dots$
    \end{itemize}
\end{axiom}

From the axioms, we can derive several useful rules for calculating probabilities of joint events, which are formed by applying set operations to individual events.

\begin{theorem}{Rules of Probability}
    \begin{itemize}
        \item \textbf{Complement Rule:} $P(A^C) = 1 - P(A)$
        \item \textbf{Addition Rule:} $P(A \cup B) = P(A) + P(B) - P(A \cap B)$
        \item \textbf{Difference Rule:} $P(A-B) = P(A) - P(A \cap B)$
        \item \textbf{Subset Rule:} If $A \subseteq B$, then $P(A) \leq P(B)$.
        \item \textbf{Empty Set Rule:} $P(\emptyset) = 0$.
    \end{itemize}
\end{theorem}

We conclude this section with a few examples illustrating the application of these rules.

\begin{example}
    A company has bid on two large construction projects. Let $A$ be the event of winning the first contract and $B$ be the event of winning the second. The company president believes that $P(A) = 0.6$, $P(B) = 0.4$, and the probability of winning both is $P(A \cap B) = 0.2$.

    \begin{enumerate}[label=(\alph*)]
        \item What is the probability that the company wins at least one contract?
        \item What is the probability that the company wins the first but not the second?
        \item What is the probability that the company wins neither contract?
        \item What is the probability that the company wins exactly one contract?
    \end{enumerate}
\end{example}

\begin{solution}
    \begin{enumerate}[label=(\alph*)]
        \item The probability of winning at least one contract is $P(A \cup B)$. Using the Addition Rule:
        \[ P(A \cup B) = P(A) + P(B) - P(A \cap B) = 0.6 + 0.4 - 0.2 = 0.8 \]
        
        \item The probability of winning the first but not the second is $P(A - B)$. Using the Difference Rule:
        \[ P(A - B) = P(A) - P(A \cap B) = 0.6 - 0.2 = 0.4 \]
        
        \item The probability of winning neither is the complement of winning at least one, $P((A \cup B)^C)$. Using the Complement Rule:
        \[ P((A \cup B)^C) = 1 - P(A \cup B) = 1 - 0.8 = 0.2 \]
        
        \item Winning exactly one contract is the symmetric difference, $P(A \triangle B)$, which can be calculated as $P(A \cup B) - P(A \cap B)$:
        \[ P(A \triangle B) = 0.8 - 0.2 = 0.6 \]
    \end{enumerate}
\end{solution}

\begin{example}
    Let $A$ be the event that it rains today and $B$ be the event that it rains tomorrow. We are given:
    \begin{itemize}
        \item $P(A) = 0.6$ (60\% chance of rain today).
        \item $P(B) = 0.5$ (50\% chance of rain tomorrow).
        \item There is a 30\% chance it does not rain either day. This means the probability of no rain today AND no rain tomorrow is $P(A^C \cap B^C) = 0.3$.
    \end{itemize}
    
    Calculate the following probabilities:
    \begin{enumerate}[label=(\alph*)]
        \item It will rain today or tomorrow.
        \item It will rain today and tomorrow.
        \item It will rain today but not tomorrow.
        \item It will rain on exactly one of the two days.
    \end{enumerate}
\end{example}
    
\begin{solution}
    \begin{enumerate}[label=(\alph*)]
        \item The probability of rain today or tomorrow is $P(A \cup B)$. We can find this using the given information about no rain. By De Morgan's Law, the event "no rain on either day" ($A^C \cap B^C$) is the complement of "rain on at least one day" ($(A \cup B)^C$).
        \begin{align*}
            P(A \cup B) &= 1 - P((A \cup B)^C) &&\text{by Complement Rule} \\
                        &= 1 - P(A^C \cap B^C) &&\text{by De Morgan's Law} \\
                        &= 1 - 0.3 = 0.7
        \end{align*}
        The probability it rains on at least one of the days is 0.7.

        \item The probability of rain today and tomorrow is $P(A \cap B)$. We use the Addition Rule, rearranged to solve for the intersection:
        \begin{align*}
            P(A \cap B) &= P(A) + P(B) - P(A \cup B) \\
                        &= 0.6 + 0.5 - 0.7 = 0.4
        \end{align*}
        The probability it rains on both days is 0.4.

        \item The probability of rain today but not tomorrow is $P(A - B)$. Using the Difference Rule:
        \begin{align*}
            P(A - B) &= P(A) - P(A \cap B) \\
                      &= 0.6 - 0.4 = 0.2
        \end{align*}
        The probability it rains today but not tomorrow is 0.2.
        
        \item The probability of rain on exactly one day is the symmetric difference, $P(A \triangle B)$. This is the probability of (rain today and not tomorrow) OR (rain tomorrow and not today). Since these two events are disjoint, we can add their probabilities: $P(A-B) + P(B-A)$.
        
        First, we find $P(B-A)$:
        \[ P(B - A) = P(B) - P(A \cap B) = 0.5 - 0.4 = 0.1 \]
        
        Now, add the two disjoint probabilities:
        \[ P(A \triangle B) = P(A-B) + P(B-A) = 0.2 + 0.1 = 0.3 \]
        The probability it rains on exactly one of the days is 0.3.
    \end{enumerate}
\end{solution}